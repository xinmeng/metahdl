Before discussing syntax details of \mhdl{}, several basic concepts
should be introduced, including terminologies about \cfg{} (CFG) and
philosophy of \mhdl{}. These preliminary information will help
understand later chapters. Readers who already familiar with CFG
and \vperl{} can skip this chapter.

\section{Grammar}
\mhdl{} is defined using CFG, which means one or more \emph{syntactic groupings} 
are defined and constructed from their parts via rules.  For example,
one rule for making an expression might be, ``An expression can be
made of a minus sign and another expression''. Another would be, ``An
expression can be an integer''.  As you can see, rules are often
recursive, but there must be at least one rule which leads out of the
recursion.

In the formal grammatical rules for a language, each kind of syntactic
unit or grouping is named by a \emph{symbol}. Those which are built by
grouping smaller constructs according to grammatical rules are
called \emph{nonterminal} symbols; those which can't be subdivided are
called \emph{terminal symbols} or
\emph{token types}. We call a piece of input corresponding to a single terminal
symbol a \emph{token}, and a piece corresponding to a single
nonterminal symbol a \emph{grouping}.

We can use the \vlog{} as an example of what symbols, terminal and
nonterminal, mean. The tokens of \vlog{} are identifiers, constants
(numeric and string), and the various keywords, arithmetic operators
and punctuation marks. So the terminal symbols of a grammar
for \vlog{} include `identifier', `number', `string', plus one symbol
for each keyword, operator or punctuation mark: `if', `return',
`const', `static', `int', `char', `plus-sign', `open-brace',
`close-brace', `comma' and many more. (These tokens can be subdivided
into characters, but that is a matter of lexicography, not grammar.)
Here is a simple \vlog{} combinational block subdivided into tokens:
\begin{lstlisting}[caption={\vlog{} example for describing CFG}]
always_comb begin       // keyword `always_comb', keyword `begin'
  if ( output_gated )   // keyword `if', open-paren, identifier, close-paren
    ctrl = 1'b0;        // identifier, equal-sign, based_number
  else                  // keyword `else'
    ctrl = i1 & i2;     // identifier, equal-sign, identifier, logic-AND, identifier
end                     // keyword `end'
\end{lstlisting}

The syntactic groupings of \vlog{} include the expression, the
statement, the declaration, and etc. These are represented in the
grammar of \vlog{} by nonterminal symbols `expression', `statement',
`declaration'. The full grammar uses dozens of additional language
constructs, each with its own nonterminal symbol, in order to express
the meanings of them. The example above is a combinational logic; it
contains one `if-else' statement. In the statement, `i1' and `i2' are
expressions, and so is `i1 \& i2'.

Each nonterminal symbol must have grammatical rules showing how it is
made out of simpler constructs. For example, one kind of \vlog{}
statement is the \texttt{assign} statement; this would be described
with a grammar rule which reads informally as follows:
\begin{quote}
An `assign statement' can be made of an `identifier', an `equal-sign',
an `expressions' and a `semicolon'.
\end{quote}


\section{\mhdl{} Philosophy}
RTL designs are not like other programming, there are few local
variables.  In additional to physical resources occupation semantics,
RTL variables also represent nets or connections. Physical elements
are connected via variables. Normally, there is no floating net inside
modules, which means every net has sources and sink. If a net has no
source, it should be module input port. If it has no sink, it should
be output port.  If it has both sources and sink, it is most probably
an internal net. These are basic rules of port inference
in \mhdlc{}. Designers can override these rules by adding explicit
port declarations.

Module ports is automatically inferred by compiler, but designers
can always command compiler to perform port validation (or port
checking) against designers' explicit declaration. In this situation,
golden ports are declared by designers, compiler compares the ports
inferred and ports declared by designers, any missing or newly
emerging ports are reported via warning.


In \mhdl{} world, there are \emph{only} four types of building block
in synthesizable RTL designs: combinational logic, sequential logic
(mostly Flip-Flop), module instantiation and FSM\footnote{ FSM is
essentially a mixture of combinational and sequential logic, Because
it is so commonly used, we promote it to basic structure.}, they are
called \emph{code block}\index{code block} in the rest of this
document. Any modules -- no matter how complex it is -- can be
decomposed to these four structures. Module is only a physical
resources wrapper with parameters to be overridden upon instantiation.

\mhdl{} RTL design is a process in which designers describe functionalities 
using code blocks; \mhdlc{} connects nets with same name and infers
ports according to designers' declarations. Parameters are recognized
and ports/nets are parameterized automatically.

\mhdl{} also allows designers to embed script like flexible code configuration
settings in RTL description in a reuse oriented design. Module logic
can be fine-grained tuned before translating \mhdl{} to \sv{}. Ports
and variable declarations are \emph{dynamically} updated according to
logic configuration.
