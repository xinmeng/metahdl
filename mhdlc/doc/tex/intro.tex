\mhdl{} is a \hdl{} (HDL) aims at synthesizable digital VLSI designs (commonly 
known as \rtl{} (RTL) designs. \mhdl{} selectively inherits \sv{}
syntax, eliminates unnecessary variants, extends existing
synthesizable language structures and adds new grammars to simplify
RTL coding. Designers will find it quite intuitive and flexible when
using \mhdl{}. A compiler named \mhdlc{}\index{\mhdlc{}} is
implemented to translate \mhdl{} to \sv{}.  

\section{Features}
\begin{enumerate}
\item Comprehensive Preprocessor
\item Flexible declarations
\item Port inference and automatic variable declarations
\item Enhanced instantiation syntax
\item New syntax for \ff{}\index{Flip-Flop} (FF) and \fsm{}\index{Finite State Machine} (FSM)
\item Parameter tracing
\item Automatic dependency resolving
\item Lightweight lint checking
\item Independent Verilog Parser to support IP integration
\item Rich user control syntax
\item Re-indent the generated \sv{}
\end{enumerate}

\section{Document Organization}
In the reset of this manual, \mhdl{} syntax and usage will be
documented in detail, following is the organization of this document:
\begin{itemize}
\item \autoref{sec:basic concepts} gives many basic and important concepts about
language and RTL design, all readers are expected to read this chapter
carefully, otherwise, later chapters are difficult to understand.
\item \autoref{sec:syntax} gives major syntax explanations and sample codes. 
After reading this chapter, readers can develop complex chips with
powerful capabilities provided by \mhdl{}.
\item \autoref{sec:pp} describes preprocessor in \mhdlc{} and its directives. Designers can 
achieve script like code configurations by using this build-in
preprocessor, instead of writing dozens of one-time scripts.
\item \autoref{sec:user control} lists various user control syntax that alters compiler execution. 
\item \autoref{sec:flow} describes mechanism and operation flow of \mhdlc{}.
\item \autoref{sec:command line opt} documents all command line options accepted by \mhdlc{}.
\item \autoref{sec:for vperl} provides additional information for those who originally use \vperl{} 
for daily coding. Differences with \vperl{} are summarized there.
\item \autoref{sec:bnf} is the complete formal syntax of \mhdl{}.
\end{itemize}

\section{Download}
\mhdlc{} is not available publically right now (hopefully next few monthes),
but anyone can goto \href{http://metahdl.tech.officelive.com/download.aspx}{MetaHDL Code Request} to register and get invitation. Registered users will receive
complete source code package and be notified on any update and new releases. 

This document is built from \LaTeX{} sources which is distributed with \mhdlc{}
source code. You can refer to \hyperref[build time]{build time} to check the 
version of this document. Latest version in PDF format can be checked out 
from \href{http://code.google.com/p/metahdl/source/checkout}{SVN}. 

\section{Bug Report}
If you find any bug of \mhdlc{}, ambiguous contents or typo in this
document, please contact author via
\href{mailto:\DocAuthor}{\DocAuthor}, thanks!



