\gls{mhdl} is an \gls{hdl} aims at synthesizable digital VLSI designs (commonly 
known as \gls{rtl} designs). \gls{mhdl} selectively inherits \gls{sv}
syntax, eliminates unnecessary variants, extends existing
synthesizable language structures and adds new grammars to simplify
RTL coding. Designers will find it quite intuitive and flexible when
using \gls{mhdl}. A compiler named \gls{mhdlc} is
implemented to translate \gls{mhdl} to \gls{sv} or \gls{v}.  

\section{Features}
\begin{enumerate}
\item Comprehensive Preprocessor
\item Flexible declarations
\item Port inference and automatic variable declarations
\item Enhanced instantiation syntax
\item New syntax for \gls{ff} and \gls{fsm}
\item Parameter tracing
\item Automatic dependency resolving
\item Lightweight lint checking
\item Independent Verilog Parser to support IP integration
\item Rich user control syntax
\item Re-indent the generated \gls{sv}/\gls{v}
\end{enumerate}

\section{Document Organization}
In the reset of this manual, \gls{mhdl} syntax and usage will be
documented in detail, following is the organization of this document:
\begin{itemize}
\item \autoref{sec:basic concepts} gives many basic and important concepts
behind \gls{mhdl}. All readers are expected to read this chapter
carefully, otherwise, later chapters are difficult to understand.
\item \autoref{sec:syntax} gives major syntax explanations and sample codes. 
After reading this chapter, readers can develop complex chips with
powerful capabilities provided by \gls{mhdl}.
\item \autoref{sec:pp} describes preprocessor in \gls{mhdlc} and
  support directives. Designers can 
achieve script-like code configurations by using this build-in
preprocessor, instead of writing dozens of one-time scripts.
\item \autoref{sec:user control} lists various user control
  variables that alter compiler execution. 
\item \autoref{sec:flow} describes mechanism and operation flow of \gls{mhdlc}.
\item \autoref{sec:command line opt} documents all 
  command line options accepted by \gls{mhdlc}.
\item \autoref{sec:for vperl} provides additional information for those who originally use \gls{vperl}\cite{Yan2004}
for daily coding. Differences with \gls{vperl} are summarized there.
\item \autoref{sec:bnf} is the complete formal syntax of \gls{mhdl}.
\end{itemize}

\section{Download}
\gls{mhdlc} is publically available at \url{http://metahdl.googlecode.com}.
This document is built from \LaTeX{} sources which are distributed with \gls{mhdlc}
source code. 

\section{Bug Report}
If you find any bug of \gls{mhdlc}, ambiguous contents or typo in this
document, please contact author via
\href{mailto:\DocAuthor}{\DocAuthor}, thanks!



