FSM\index{FSM} in conventional RTL design requires many
constant/parameter definitions to make code readable. But these
definitions are hard to maintain during develop iteration, especially
for one-hot encoded FSM. \mhdl{} introduces \emph{symbol based} FSM
programming paradigm that liberates designers from such frustrated
situation.

fsm\_block\footnote{If you still don't know what is fsm\_block, I
guess you need to print out
\autoref{sec:bnf} and look up non-terminals in it when spotted.}
is enclosed by keywords `fsm' and `endfsm'. Note that `fsm'
line \emph{must} end with semi-colon, just as \sv{} `sequence' or
`property' blocks. `fsm' is followed by three identifiers: FSM name,
clock name, and reset name. FSM name is used as based name of state
register, `\_cs' and `\_ns' suffix are appended to FSM name to create
current state register and next state next state register,
respectively. clock and reset names are used in sequential block of
FSM, which resets state register and perform current state
refreshing. clock and reset names can be omitted together, and default
name `clock' and `reset\_n' will be used. State transition explicitly
stated by `goto' keyword, instead of next state assignment.

Symbol based FSM programming allows designers to code FSM using
state names, one-hot state encodings are automatically generated
by \mhdlc{}. Constant definitions are generated according to state
names to improve code readability. To help designers eliminate state
name typo, \mhdlc{} will build a directed graph representing state
transition during parsing, to check the connectivity of every state. Dead
states and unreachable states are reported to designers for
confirmation. \autoref{lst:fsm in mhdl} is \mhdl{} FSM description,
\autoref{lst:fsm in sv} is the corresponding \sv{} description, including
constant definition. 

\begin{minipage}[t]{.45\textwidth}
\begin{lstlisting}[caption={FSM in \mhdl}, label={lst:fsm in mhdl}]
fsm cmdrx, clk, rst_n;

  cm_pim_ack = 1'b0;
  
  IDLE: begin
    if ( pim_cm_req ) begin
      cm_pim_ack = 1'b1;
      goto DATA;
    end
    else begin
      goto IDLE;
    end
  end

  DATA: begin
    cm_pim_ack = 1'b1;
    if ( pim_cm_eof ) begin
      cm_pim_ack = 1'b0;
      goto IDLE;
    end
    else begin
      goto DATA;
    end
  end

endfsm
\end{lstlisting}
\end{minipage}
\hspace{1ex}
\begin{minipage}[t]{.5\textwidth}
\begin{lstlisting}[caption={FSM in \sv}, label={lst:fsm in sv}]
// other declarations...
const logic [1:0] DATA = 2'b10;
const logic [1:0] IDLE = 2'b01;
const int _DATA_ = 1;
const int _IDLE_ = 0;

// Sequential part of 
// FSM /tmp/xin_meng/mhdlc/test/a.mhdl:1.0-25.5
// /tmp/xin_meng/mhdlc/test/a.mhdl:1.0-25.5
always_ff @(posedge clk or negedge rst_n)
  if (~rst_n) begin
    cmdrx_cs <= IDLE;
  end
  else begin
    cmdrx_cs <= cmdrx_ns;
  end

// Combnational part of
// FSM /tmp/xin_meng/mhdlc/test/a.mhdl:1.0-25.5
// /tmp/xin_meng/mhdlc/test/a.mhdl:1.0-25.5
always_comb begin
  cm_pim_ack = 1'b0;
  unique case ( 1'b1 )
    cmdrx_cs[_IDLE_] : begin
      if ( pim_cm_req ) begin
        cm_pim_ack = 1'b1;
        cmdrx_ns = DATA;
      end
      else begin
        cmdrx_ns = IDLE;
      end
    end

    cmdrx_cs[_DATA_] : begin
      cm_pim_ack = 1'b1;
      if ( pim_cm_eof ) begin
        cm_pim_ack = 1'b0;
        cmdrx_ns = IDLE;
      end
      else begin
        cmdrx_ns = DATA;
      end
    end

    default: begin
      cmdrx_ns = 2'hX;
    end
  endcase
end
\end{lstlisting}
\end{minipage}

As shown in \autoref{lst:fsm in sv}, `fsm\_block' is expanded to two blocks: sequential 
and combinational. The former resets state register, the latter calculates next state and 
controls output. Combinational part of FSM is implemented in `unique case' statement, a bunch 
of constants are defined to hold state value and hot bit index. 
