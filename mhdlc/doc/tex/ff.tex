\mhdl{} supports two Flip-Flop descriptions: 
one is conventional \texttt{always} 
block using keywords `posedge' and `negedge' to denote Flip-Flop
description, the other is newly added syntax using `ff' and `endff' keywords.
\autoref{lst:new ff} demonstrates new Flip-Flop syntax, 
\autoref{lst:legacy ff} is the equivalence conventional syntax, both of 
them are legal in \mhdlc{}. 

\begin{minipage}[t]{.45\textwidth}
\begin{lstlisting}[caption={\mhdl{} FF syntax}, label={lst:new ff}]
ff clk, rst_n;
  a_ff, i1 ? c : d, 1'b0;
  b_ff, b & c;
endff

// data path does not need reset
ff clk; 
  data_ff[63:0], data[63:0];
endff
\end{lstlisting}
\end{minipage}
\hspace{1ex}
\begin{minipage}[t]{.5\textwidth}
\begin{lstlisting}[caption={Legacy FF syntax},
label={lst:legacy ff}]
// OK, accepted
always_ff @ (posedge clk or negedge rst_n )
  if ( ~rst_n ) begin
    a_ff <= 1'b0;
  end
  else begin
    a_ff <= i1 ? c: d;
    b_ff <= b & c;
  end

// OK, accepted
// data path does not need reset
always_ff @ (posedge clk)
  data_ff[63:0] <= data[63:0];
\end{lstlisting}
\end{minipage}

In \mhdl{} new FF syntax, keyword `ff' is followed by two optional 
identifiers: one is clock name, the other is reset name. If reset
name is omitted, there is no reset clause in \texttt{always\_ff} 
block. If both of them are omitted, default name `clock' and `reset\_n' 
are used, both reset clause and value refresh clause are generated. 

`ff\_item'\footnote{Maybe you need \autoref{sec:bnf} if you don't know what I'm saying.} 
consists of three parts: FF name, source value expression, and
reset value. Reset value is optional, when it is omitted, corresponding reset behavior 
is not generated in \sv{}.


