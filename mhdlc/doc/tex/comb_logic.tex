Tow code blocks can be used to describe combinational logic:
\begin{itemize}
\item \texttt{always\_comb} code block
\item \texttt{assign} statement
\end{itemize}
Different from \vlog{} or \vlog{} 2000, \sv{} introduces \texttt{always\_comb}
keywords and eliminates sensitivity list, which simplifies combinational logic 
coding a lot. \mhdl{} \emph{only} supports \texttt{always\_comb} style procedure
assignment, legacy \vlog{} or \vlog{} 2000 with sensitivity list are not recognized. 
\mhdl{} also supports \texttt{assign} continuous assignment to describe simple combinational
logic. \autoref{lst:comb example} demonstrates legal and illegal combinational logic code
in \mhdl{}.
\begin{lstlisting}[caption={Combinational logic examples}, label={lst:comb example}]
// /*\textcolor{green}{OK, accepted} */
always_comb 
  if ( enabled ) 
    o1 = i1 | i2 | i3;
  else 
    o1 = 1'b0;

// /*\textcolor{green}{OK, accepted}*/
assign o2 = cond ? i1 : i2;

// /*\textcolor{red}{Illegal, wrong!!}*/
// /*\textcolor{red}{conventional Verilog}*/
always @( i1 or i2 or i3 )
  if ( enabled ) 
    o1 = i1 | i2 | i3;
  else 
    o1 = 1'b0;

// /*\textcolor{red}{Illegal, wrong!!}*/
// /*\textcolor{red}{Verilog 2000 is not accepted, neither}*/
always @(*)
  if ( enabled ) 
    o1 = i1 | i2 | i3;
  else 
    o1 = 1'b0;
\end{lstlisting}
