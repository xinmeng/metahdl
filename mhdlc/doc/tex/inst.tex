\vlog{} module instantiation syntax is extended in \mhdl{}, BNF is:
\begin{quote}
\mbox{inst\_block $::=$\hspace{1ex} \textbf{\textcolor{red}{ID}} parameter\_rule instance\_name connection\_spec \textbf{\textcolor{red}{;}}}
\end{quote}
Where `ID' is the module name to be instantiated,
parameter\_rule, instance\_name and connection\_spec are all optional. 
If no instance name specified, prefix `x\_' is added to module name to create
instance name. 

parameter\_rule specifies parameter override. In addition to \vlog{}
positioned override,
\emph{named parameter override} is added. Designers can explicitly specify which parameter 
should be set, rather than list all magic numbers.

In addition to \vlog{} connection syntax, connection\_spec supports
prefix, suffix and regular expression connection rules, which save a
lot efforts in IP integration and top level integration.  Note that
prefix, suffix and regular expression rules are cumulative and
applicable to all module ports, rule execution sequence is the
sequence they appear.

\autoref{lst:moda in mhdl} is the module to be instantiated.
\autoref{lst:modb in mhdl} instantiates \texttt{moda} several times, 
pay attention to the instance at line \autoref{ln:inst 2},
prefix and suffix rules take \emph{cumulative} effects
on \texttt{x2\_moda}.
\autoref{lst:modb in sv} is the generated \sv{}. 

\begin{minipage}[t]{.45\textwidth}
\begin{minipage}[t]{\textwidth}
\begin{lstlisting}[caption={Wrapper Module in \mhdl},label={lst:modb in mhdl}]
// simplest instantiation
moda;


// prefix rule
moda x1_moda ( x1_ +);

// suffix rule
// after prefix rule
moda x2_moda ( x2_ + , /*\label{ln:inst 2}*/
               + _22);


// Perl compatible regexp
moda x3_moda ( "s/o/out/g", 
               "s/i/in/g" );
\end{lstlisting}
\end{minipage}

\begin{minipage}[t]{\textwidth}
\begin{lstlisting}[caption={Module Template}, label={lst:moda in mhdl}]
module moda (
  i1, 
  i2, 
  i3, 
  i4, 
  i5, 
  i6, 
  o1, 
  o2);


input i1;
input i2;
input i3;
input i4;
input i5;
input i6;
output o1;
output [1:0] o2;

endmodule
\end{lstlisting}
\end{minipage}
\end{minipage}
\hspace{1ex}
\begin{minipage}[t]{.5\textwidth}
\begin{lstlisting}[caption={Wrapper module in \sv{}}, label={lst:modb in sv}]
// declarations...

// /tmp/xin_meng/mhdlc/test/modb.mhdl:2.0-4
moda x_moda (
                .o1 (o1),
                .i1 (i1),
                .i2 (i2),
                .o2 (o2),
                .i3 (i3),
                .i4 (i4),
                .i5 (i5),
                .i6 (i6)
            );

// /tmp/xin_meng/mhdlc/test/modb.mhdl:6.0-21
moda x1_moda (
                 .o1 (x1_o1),
                 .i1 (x1_i1),
                 .i2 (x1_i2),
                 .o2 (x1_o2),
                 .i3 (x1_i3),
                 .i4 (x1_i4),
                 .i5 (x1_i5),
                 .i6 (x1_i6)
             );

// /tmp/xin_meng/mhdlc/test/modb.mhdl:10.0-11.21
moda x2_moda (
                 .o1 (x2_o1_22),
                 .i1 (x2_i1_22),
                 .i2 (x2_i2_22),
                 .o2 (x2_o2_22),
                 .i3 (x2_i3_22),
                 .i4 (x2_i4_22),
                 .i5 (x2_i5_22),
                 .i6 (x2_i6_22)
             );

// /tmp/xin_meng/mhdlc/test/modb.mhdl:15.0-16.27
moda x3_moda (
                 .o1 (out1),
                 .i1 (in1),
                 .i2 (in2),
                 .o2 (out2),
                 .i3 (in3),
                 .i4 (in4),
                 .i5 (in5),
                 .i6 (in6)
             );
\end{lstlisting}
\end{minipage}
