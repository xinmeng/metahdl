\gls{mhdl} syntax borns from \gls{sv}. It selectively
 inherits synthesizable syntax of \gls{sv}, eliminates unnecessary
variants, extends module instantiation syntax, add new syntax for
Flip-Flop and FSM. \gls{v} or \gls{sv} designers will find it quite
intuitive to use \gls{mhdl} syntax. In the rest of this chapter,
major syntax are presented with examples, refer to \autoref{sec:bnf} for complete syntax.

\section{Combinational Logic}
There are two and only two types of \gls{code block} in 
\gls*{mhdl} for coding combinational logic:
\begin{enumerate}
\item \mhdl{assign} statement
\item \mhdl{always_comb} statement
\end{enumerate}
Both syntax are same in \gls*{sv}. 

\section{Sequential Logic}
There are two types of \gls{code block} in \gls*{mhdl} for coding
sequential logic:
\begin{enumerate}
\item \mhdl{always_ff @()} statement, which is same in \gls*{sv}
\item \mhdl{ff} - \mhdl{endff} block, which is introduec by \gls*{mhdl}
\end{enumerate}
The troditional \gls*{sv} syntax is good except its redundancy: \gls{ff}
variable appears twice (even more) in different clause of \mhdl{if}-\mhdl{else}
branches. For a multi-bit vector variable, such redundancy could lead 
to typo and width mismatch among all branches. 

For well-coded \gls{ff}, combinational part of the \gls{ff} sources should be
coded in a separate \gls{code block}, so the \mhdl{if}-\mhdl{else} branches 
can be reduced. \gls{mhdl} provides a new \mhdl{ff}-\mhdl{endff} \gls{code block}
to such reduced \gls*{ff} coding. Such new syntax can eliminate redundancy. 
\gls{ff} code in follow two forms are equivalent. 
\begin{mhdle}
// troditional sequential block
always_ff @ (posedge clk or negedge rst_n)
  if (!rst_n)
     a_ff <= 1'b0;
  else 
     a_ff <= a;

// MetaHDL new sequential block
ff; 
  a_ff, a, 1'b0;
endff
\end{mhdle}

\mhdl{ff} block can specify clock and reset signal name, however, they are 
usually ommitted to further reduce redundancy. Each line in \mhdl{ff} block
describes a \gls{ff}. A line has three element:
\begin{enumerate}
\item \gls{ff} variable name, \mhdl{a_ff} here. 
\item An expression containing the logic to update the \gls{ff}, \mhdl{a} here. 
  Any expression defined in \autoref{sec:bnf} are allowed here. 
\item An constant as reset value. It is optional. If not reset value provided, 
  \gls{ff} variable will not be reset. 
\end{enumerate}
Here is example:

\begin{minipage}[t]{.48\textwidth}
\begin{mhdle}[caption={MetaHDL FF syntax}]
ff clk_a, clk_a_rst_n;
   a_ff, a, 1'b0;
   b_ff, b;
   c_ff, a_ff & b_ff, 1'b0;
endff
\end{mhdle}
\end{minipage}
\hfill
\begin{minipage}[t]{.48\textwidth}
\begin{mhdle}[caption={Generated SystemVerilog}]
always_ff @(posedge clk_a or negedge clk_a_rst_n)
  if (clk_a_rst_n)
    a_ff <= 1'b0;
    c_ff <= 1'b0;
  else begin
     a_ff <= a;
     b_ff <= b;
     c_ff <= a_ff & b_ff;
  end
\end{mhdle}
\end{minipage}

\section{FSM}
\gls{fsm} in conventional RTL design requires many
constant/parameter definitions to improve code readability. But these
definitions are hard to maintain during develop iteration, especially
for one-hot encoded FSM. \gls{mhdl} introduces \emph{symbol based} FSM
programming paradigm that liberates designers from such frustrated
situation.

\gls{fsm} \gls{code block}
is enclosed by keywords \mhdl{fsm}/\mhdl{fsm_nc} and \mhdl{endfsm}.
\mhdl{fsm} is followed by three identifiers: 
\begin{enumerate}
\item FSM name, which is mandatory.
\item clock signal name, which is optional.
\item reset signal name, which is optional, too. 
\end{enumerate}
FSM name is used as based name of state
register, \mhdl{_cs} and \mhdl{_ns} suffix are appended to FSM name to create
current state register and next state next state register,
respectively. clock and reset signal names are used in sequential block of
FSM, which resets state register and perform current state
refreshing. clock and reset names can be omitted together, and default
name \mhdl{clock} and \mhdl{reset\_n} will be used. State transition is explicitly
stated by \mhdl{goto} keyword, instead of next state assignment.

Symbol based FSM programming allows designers to code FSM using
state names, one-hot state encodings are automatically generated
by \gls{mhdlc}. Constant definitions are generated according to state
names to improve code readability. To help designers eliminate state
name typo, \gls{mhdlc} will build a \emph{Directed Graph} representing state
transition during parsing, to check the connectivity of every state. Dead
states and unreachable states are reported to designers for
confirmation. \autoref{lst:fsm in mhdl} is \gls{mhdl} FSM description,
\autoref{lst:fsm in sv} is the corresponding \gls{sv} description, including
constant definition. 

\begin{figure}[pbt]
\begin{minipage}[t]{.48\textwidth}
\begin{mhdle}[caption={FSM in MetaHDL}, label={lst:fsm in mhdl}]
fsm cmdrx, clk, rst_n;
  cm_pim_ack = 1'b0;
  
  IDLE: begin
    if ( pim_cm_req ) begin
      cm_pim_ack = 1'b1;
      goto DATA;
    end
    else begin
      goto IDLE;
    end
  end

  DATA: begin
    cm_pim_ack = 1'b1;
    if ( pim_cm_eof ) begin
      cm_pim_ack = 1'b0;
      goto IDLE;
    end
    else begin
      goto DATA;
    end
  end

endfsm
\end{mhdle}
\end{minipage}
\hfill
\begin{minipage}[t]{.48\textwidth}
\begin{mhdle}[caption={Generated FSM in SystemVerilog}, label={lst:fsm in sv}]
// other declarations...
const logic [1:0] DATA = 2'b10;
const logic [1:0] IDLE = 2'b01;
const int _DATA_ = 1;
const int _IDLE_ = 0;

// Sequential part of  cmdrx
always_ff @(posedge clk or negedge rst_n)
  if (~rst_n) begin
    cmdrx_cs <= IDLE;
  end
  else begin
    cmdrx_cs <= cmdrx_ns;
  end

// Combnational part of cmdrx
always_comb begin
  cm_pim_ack = 1'b0;
  unique case ( 1'b1 )
    cmdrx_cs[_IDLE_] : begin
      if ( pim_cm_req ) begin
        cm_pim_ack = 1'b1;
        cmdrx_ns = DATA;
      end
      else begin
        cmdrx_ns = IDLE;
      end
    end

    cmdrx_cs[_DATA_] : begin
      cm_pim_ack = 1'b1;
      if ( pim_cm_eof ) begin
        cm_pim_ack = 1'b0;
        cmdrx_ns = IDLE;
      end
      else begin
        cmdrx_ns = DATA;
      end
    end

    default: begin
      cmdrx_ns = 2'hX;
    end
  endcase
end
\end{mhdle}
\end{minipage}
\end{figure}

Different of \mhdl{fsm} and \mhdl{fsm_nc} is Verilog generated 
from \mhdl{fsm_nc} block will not contain the sequential block. 
That means designers have to manually code the sequential block. 
This is expecially designed for FSM with synchronous reset. 
\textbf{Note} that the manual crafted sequential block \emph{must}
come after the \mhdl{fsm_nc} block, because \mhdl{*_ns} and \mhdl{*_cs}
signals are only accessible after \mhdl{fsm_nc} block is parsed. 

\section{Module Instantiation}
\gls{sv} module instantiation syntax is extended in \gls{mhdl}, BNF is
shown in \autoref{sec:bnf}, start from non-terminal ``inst\_block''.
Features of \gls{mhdl} instantiation syntax are highlighted below:
\begin{enumerate}
\item Instance name is optional. Default instance name is prefix \mhdl{x_}
concatenate with module name. 
\item Port connection is optional. Default behavior is to connect ports 
  to net with identical name. 
\item Prefix and/or Suffix rule is allowed in port connection (see example below). 
\item Regular expression rule is allowed in port connection (see example below).
\end{enumerate}
\begin{figure}[ptb]
  \begin{minipage}[t]{.48\textwidth}
    \begin{mhdle}[caption={\texttt{moda.mhdl}}]
input i1;
input i2;
output o1;
output [1:0] o2;
    \end{mhdle}
    \begin{mhdle}[caption={Instantiation in MetaHDL}]
// simplest instantiation
moda;

// prefix rule
moda x1_moda ( x1_ +);

// suffix rule
// after prefix rule
moda x2_moda ( x2_ + , /*\label{ln:inst 2}*/
               + _22);


// Perl compatible regexp
moda x3_moda ( "s/o/out/g", 
               "s/i/in/g" );
    \end{mhdle}
  \end{minipage}
  \hfill
  \begin{minipage}[t]{.48\textwidth}
    \begin{mhdle}[caption={Generated SystemVerilog}]
moda x_moda (
                .i1 (i1),
                .i2 (i2),
                .o1 (o1),
                .o2 (o2)
            );

moda x1_moda (
                 .i1 (x1_i1),
                 .i2 (x1_i2),
                 .o1 (x1_o1),
                 .o2 (x1_o2)
             );

moda x2_moda (
                 .i1 (x2_i1_22),
                 .i2 (x2_i2_22),
                 .o1 (x2_o1_22),
                 .o2 (x2_o2_22)
             );

moda x3_moda (
                 .i1 (in1),
                 .i2 (in2),
                 .o1 (out1),
                 .o2 (out2)
             );
    \end{mhdle}
  \end{minipage}
\end{figure}


\section{Parameter Tracing}
\gls{mhdl} enables designers to creates parameterized module in two ways: 
\begin{itemize}
\item Write parameterized module from draft.
\item Build parameterized module from existing parameterized modules. 
\end{itemize}
Designers declare parameters, and use them in ports or net index. \mhdlc{} 
will automatically parameterize ports in generated declarations. If a module 
to be instantiated is a parameterized module, \mhdlc{} can trace parameter usage
in port connections and automatically parameterize wrapper module. 
\autoref{lst:modc in mhdl}, \autoref{lst:modc in sv}, \autoref{lst:wrapper in mhdl} and 
\autoref{lst:wrapper in sv} demonstrate a example parameter tracing.

\begin{figure}[ptb]
  \begin{minipage}[t]{.48\textwidth}
    \begin{mhdle}[caption={Inner module: \texttt{modc} in \gls{mhdl}}, label={lst:modc in mhdl}]
parameter A = 4;
parameter B = 5; 
parameter C = A + B;

assign o1[C-1:0] = {~i1[A-1:0], i2[B-1:0]};
    \end{mhdle}
    \begin{mhdle}[caption={Generated \texttt{modc} in SystemVerilog}, label={lst:modc in sv}]
module modc (
  i1, 
  i2, 
  o1);

parameter A = 4;
parameter B = 5;
parameter C = 4 + 5;

input [A - 1:0] i1;
input [B - 1:0] i2;
output [C - 1:0] o1;

logic [A - 1:0] i1;
logic [B - 1:0] i2;
logic [C - 1:0] o1;

assign o1[C - 1:0] = {~i1[A - 1:0], i2[B - 1:0]};
endmodule
    \end{mhdle}
    \begin{mhdle}[caption={Wrapper Module in MetaHDL}, label={lst:wrapper in mhdl}]
parameter SETA = 8,
          SETB = 9;


modc #( .A(2) ) x0_modc ( x0_ + );

modc #( SETA, SETB ) x1_modc ( x1_ + );

modc #( .A(SETA) ) x2_modc (x2_ +,
                           .o1 (x2_o1[10:0]));
    \end{mhdle}
  \end{minipage}
  \hfill
  \begin{minipage}[t]{.48\textwidth}
    \begin{mhdle}[caption={Generated wrapper module in SystemVerilog}, label={lst:wrapper in sv}]
module modwrapper (
  x0_i1, 
  x0_i2, 
  x0_o1, 
  x1_i1, 
  x1_i2, 
  x1_o1, 
  x2_i1, 
  x2_i2, 
  x2_o1);

parameter SETA = 8;
parameter SETB = 9;

input   [1   :0]  x0_i1;
input   [4   :0]  x0_i2;
output  [6   :0]  x0_o1;
input   [SETA - 1:0]  x1_i1;
input   [SETB - 1:0]  x1_i2;
output  [SETA + SETB - 1:0]  x1_o1;
input   [SETA - 1:0]  x2_i1;
input   [4   :0]  x2_i2;
output  [10  :0]  x2_o1;

logic   [1   :0]  x0_i1;
logic   [4   :0]  x0_i2;
logic   [6   :0]  x0_o1;
logic   [SETA - 1:0]  x1_i1;
logic   [SETB - 1:0]  x1_i2;
logic   [SETA + SETB - 1:0]  x1_o1;
logic   [SETA - 1:0]  x2_i1;
logic   [4   :0]  x2_i2;
logic   [10  :0]  x2_o1;

modc #(
       .A( 2 ),
       .B( 5 ),
       .C( 2 + 5 )	 
      ) x0_modc (
                 .i1 (x0_i1),
                 .i2 (x0_i2),
                 .o1 (x0_o1)
                );

modc #(
       .A( SETA ),
       .B( SETB ),
       .C( SETA + SETB )	 
      ) x1_modc (
                 .i1 (x1_i1),
                 .i2 (x1_i2),
                 .o1 (x1_o1)
                );

modc #(
       .A( SETA ),
       .B( 5 ),
       .C( SETA + 5 )	 
      ) x2_modc (
                 .i1 (x2_i1),
                 .i2 (x2_i2),
                 .o1 (x2_o1[10:0])
                );

endmodule
    \end{mhdle}
  \end{minipage}
\end{figure}

\section{Optional Declaration}
Declaration in \gls{v} and \gls{sv} is mandatory, but in \gls{mhdl} is optional.
\gls{mhdlc} can automatcially infer width, port direction, and variable type 
from a well designed synthesizable RTL code. But in some cases, designers want
to override the inference results. This can be done by declaration statements.
Usually,  declaration is used in follow sceanrios:
\begin{enumerate}
\item A parameterized port/net/reg, such as \mhdl{assign a[A-1:0] = b[A-1:0];}
\item Force port direction, such as \mhdl{input a; output b, nonport c;}
\item 2 dimensional array, such as \mhdl{reg [31:0] a [15:0];}
\item Integer iteration variable used in \mhdl{for} statement, such as \mhdl{int i; for (i=0;i<32;i=i+1)     a[i] = b[31-i];}
\end{enumerate}

\section{Runtime Compiler Control}
