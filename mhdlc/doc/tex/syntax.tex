\mhdl{} syntax born from \sv{}. It selectively
 inherits synthesizable syntax of \sv{}, eliminates unnecessary
variants, extends module instantiation syntax, add new syntax for
Flip-Flop and FSM. \vlog{} or \sv{} designers will find it quite
intuitive to use \mhdl{} syntax. In the rest of this chapter,
major syntax are presented with examples, refer to \autoref{sec:bnf} for complete syntax.

\section{Expression and Statement}Just as \vlog{}, \sv{} or C/C++, \emph{expression}\index{expression}
is the bottom level language elements, it
constructs \emph{statement}\index{statement}, statement then
constructs code blocks\index{code block} which ultimately constructs
module. Expression is the same as that in synthesizable \vlog{}
or \sv{}. Expression is recursively defined in BNF, prime expressions
are net and constant, which leads out of the recursion.

Statement is built upon expression through assignment, conditional
control or repetition. Empty statement is also allowed in \mhdl{},
since it does not make any sense, warning is popped when empty
statement is encountered. `if-else', `case', `for' statements are
supported, `goto' statement is newly added for state transition (refer
to \autoref{sec:syntax.fsm}).

Since \mhdl{} inherits \sv{} syntax, rules for expression and
statements are same as that in \sv{}, refer to \autoref{sec:bnf} for
complete rule list. 

\section{Combinational Logic}Tow code blocks can be used to describe combinational logic:
\begin{itemize}
\item \texttt{always\_comb} code block
\item \texttt{assign} statement
\end{itemize}
Different from \vlog{} or \vlog{} 2000, \sv{} introduces \texttt{always\_comb}
keywords and eliminates sensitivity list, which simplifies combinational logic 
coding a lot. \mhdl{} \emph{only} supports \texttt{always\_comb} style procedure
assignment, legacy \vlog{} or \vlog{} 2000 with sensitivity list are not recognized. 
\mhdl{} also supports \texttt{assign} continuous assignment to describe simple combinational
logic. \autoref{lst:comb example} demonstrates legal and illegal combinational logic code
in \mhdl{}.
\begin{lstlisting}[caption={Combinational logic examples}, label={lst:comb example}]
// /*\textcolor{green}{OK, accepted} */
always_comb 
  if ( enabled ) 
    o1 = i1 | i2 | i3;
  else 
    o1 = 1'b0;

// /*\textcolor{green}{OK, accepted}*/
assign o2 = cond ? i1 : i2;

// /*\textcolor{red}{Illegal, wrong!!}*/
// /*\textcolor{red}{conventional Verilog}*/
always @( i1 or i2 or i3 )
  if ( enabled ) 
    o1 = i1 | i2 | i3;
  else 
    o1 = 1'b0;

// /*\textcolor{red}{Illegal, wrong!!}*/
// /*\textcolor{red}{Verilog 2000 is not accepted, neither}*/
always @(*)
  if ( enabled ) 
    o1 = i1 | i2 | i3;
  else 
    o1 = 1'b0;
\end{lstlisting}

\section{Flip-Flop}\mhdl{} supports two Flip-Flop descriptions: 
one is conventional \texttt{always} 
block using keywords `posedge' and `negedge' to denote Flip-Flop
description, the other is newly added syntax using `ff' and `endff' keywords.
\autoref{lst:new ff} demonstrates new Flip-Flop syntax, 
\autoref{lst:legacy ff} is the equivalence conventional syntax, both of 
them are legal in \mhdlc{}. 

\begin{minipage}[t]{.45\textwidth}
\begin{lstlisting}[caption={\mhdl{} FF syntax}, label={lst:new ff}]
ff clk, rst_n;
  a_ff, i1 ? c : d, 1'b0;
  b_ff, b & c;
endff

// data path does not need reset
ff clk; 
  data_ff[63:0], data[63:0];
endff
\end{lstlisting}
\end{minipage}
\hspace{1ex}
\begin{minipage}[t]{.5\textwidth}
\begin{lstlisting}[caption={Legacy FF syntax},
label={lst:legacy ff}]
// OK, accepted
always_ff @ (posedge clk or negedge rst_n )
  if ( ~rst_n ) begin
    a_ff <= 1'b0;
  end
  else begin
    a_ff <= i1 ? c: d;
    b_ff <= b & c;
  end

// OK, accepted
// data path does not need reset
always_ff @ (posedge clk)
  data_ff[63:0] <= data[63:0];
\end{lstlisting}
\end{minipage}

In \mhdl{} new FF syntax, keyword `ff' is followed by two optional 
identifiers: one is clock name, the other is reset name. If reset
name is omitted, there is no reset clause in \texttt{always\_ff} 
block. If both of them are omitted, default name `clock' and `reset\_n' 
are used, both reset clause and value refresh clause are generated. 

`ff\_item'\footnote{Maybe you need \autoref{sec:bnf} if you don't know what I'm saying.} 
consists of three parts: FF name, source value expression, and
reset value. Reset value is optional, when it is omitted, corresponding reset behavior 
is not generated in \sv{}.



\section{FSM}\label{sec:syntax.fsm}FSM\index{FSM} in conventional RTL design requires many
constant/parameter definitions to make code readable. But these
definitions are hard to maintain during develop iteration, especially
for one-hot encoded FSM. \mhdl{} introduces \emph{symbol based} FSM
programming paradigm that liberates designers from such frustrated
situation.

fsm\_block\footnote{If you still don't know what is fsm\_block, I
guess you need to print out
\autoref{sec:bnf} and look up non-terminals in it when spotted.}
is enclosed by keywords `fsm' and `endfsm'. Note that `fsm'
line \emph{must} end with semi-colon, just as \sv{} `sequence' or
`property' blocks. `fsm' is followed by three identifiers: FSM name,
clock name, and reset name. FSM name is used as based name of state
register, `\_cs' and `\_ns' suffix are appended to FSM name to create
current state register and next state next state register,
respectively. clock and reset names are used in sequential block of
FSM, which resets state register and perform current state
refreshing. clock and reset names can be omitted together, and default
name `clock' and `reset\_n' will be used. State transition explicitly
stated by `goto' keyword, instead of next state assignment.

Symbol based FSM programming allows designers to code FSM using
state names, one-hot state encodings are automatically generated
by \mhdlc{}. Constant definitions are generated according to state
names to improve code readability. To help designers eliminate state
name typo, \mhdlc{} will build a directed graph representing state
transition during parsing, to check the connectivity of every state. Dead
states and unreachable states are reported to designers for
confirmation. \autoref{lst:fsm in mhdl} is \mhdl{} FSM description,
\autoref{lst:fsm in sv} is the corresponding \sv{} description, including
constant definition. 

\begin{minipage}[t]{.45\textwidth}
\begin{lstlisting}[caption={FSM in \mhdl}, label={lst:fsm in mhdl}]
fsm cmdrx, clk, rst_n;

  cm_pim_ack = 1'b0;
  
  IDLE: begin
    if ( pim_cm_req ) begin
      cm_pim_ack = 1'b1;
      goto DATA;
    end
    else begin
      goto IDLE;
    end
  end

  DATA: begin
    cm_pim_ack = 1'b1;
    if ( pim_cm_eof ) begin
      cm_pim_ack = 1'b0;
      goto IDLE;
    end
    else begin
      goto DATA;
    end
  end

endfsm
\end{lstlisting}
\end{minipage}
\hspace{1ex}
\begin{minipage}[t]{.5\textwidth}
\begin{lstlisting}[caption={FSM in \sv}, label={lst:fsm in sv}]
// other declarations...
const logic [1:0] DATA = 2'b10;
const logic [1:0] IDLE = 2'b01;
const int _DATA_ = 1;
const int _IDLE_ = 0;

// Sequential part of 
// FSM /tmp/xin_meng/mhdlc/test/a.mhdl:1.0-25.5
// /tmp/xin_meng/mhdlc/test/a.mhdl:1.0-25.5
always_ff @(posedge clk or negedge rst_n)
  if (~rst_n) begin
    cmdrx_cs <= IDLE;
  end
  else begin
    cmdrx_cs <= cmdrx_ns;
  end

// Combnational part of
// FSM /tmp/xin_meng/mhdlc/test/a.mhdl:1.0-25.5
// /tmp/xin_meng/mhdlc/test/a.mhdl:1.0-25.5
always_comb begin
  cm_pim_ack = 1'b0;
  unique case ( 1'b1 )
    cmdrx_cs[_IDLE_] : begin
      if ( pim_cm_req ) begin
        cm_pim_ack = 1'b1;
        cmdrx_ns = DATA;
      end
      else begin
        cmdrx_ns = IDLE;
      end
    end

    cmdrx_cs[_DATA_] : begin
      cm_pim_ack = 1'b1;
      if ( pim_cm_eof ) begin
        cm_pim_ack = 1'b0;
        cmdrx_ns = IDLE;
      end
      else begin
        cmdrx_ns = DATA;
      end
    end

    default: begin
      cmdrx_ns = 2'hX;
    end
  endcase
end
\end{lstlisting}
\end{minipage}

As shown in \autoref{lst:fsm in sv}, `fsm\_block' is expanded to two blocks: sequential 
and combinational. The former resets state register, the latter calculates next state and 
controls output. Combinational part of FSM is implemented in `unique case' statement, a bunch 
of constants are defined to hold state value and hot bit index. 

\section{Module Instantiation}\vlog{} module instantiation syntax is extended in \mhdl{}, BNF is:
\begin{quote}
\mbox{inst\_block $::=$\hspace{1ex} \textbf{\textcolor{red}{ID}} parameter\_rule instance\_name connection\_spec \textbf{\textcolor{red}{;}}}
\end{quote}
Where `ID' is the module name to be instantiated,
parameter\_rule, instance\_name and connection\_spec are all optional. 
If no instance name specified, prefix `x\_' is added to module name to create
instance name. 

parameter\_rule specifies parameter override. In addition to \vlog{}
positioned override,
\emph{named parameter override} is added. Designers can explicitly specify which parameter 
should be set, rather than list all magic numbers.

In addition to \vlog{} connection syntax, connection\_spec supports
prefix, suffix and regular expression connection rules, which save a
lot efforts in IP integration and top level integration.  Note that
prefix, suffix and regular expression rules are cumulative and
applicable to all module ports, rule execution sequence is the
sequence they appear.

\autoref{lst:moda in mhdl} is the module to be instantiated.
\autoref{lst:modb in mhdl} instantiates \texttt{moda} several times, 
pay attention to the instance at line \autoref{ln:inst 2},
prefix and suffix rules take \emph{cumulative} effects
on \texttt{x2\_moda}.
\autoref{lst:modb in sv} is the generated \sv{}. 

\begin{minipage}[t]{.45\textwidth}
\begin{minipage}[t]{\textwidth}
\begin{lstlisting}[caption={Wrapper Module in \mhdl},label={lst:modb in mhdl}]
// simplest instantiation
moda;


// prefix rule
moda x1_moda ( x1_ +);

// suffix rule
// after prefix rule
moda x2_moda ( x2_ + , /*\label{ln:inst 2}*/
               + _22);


// Perl compatible regexp
moda x3_moda ( "s/o/out/g", 
               "s/i/in/g" );
\end{lstlisting}
\end{minipage}

\begin{minipage}[t]{\textwidth}
\begin{lstlisting}[caption={Module Template}, label={lst:moda in mhdl}]
module moda (
  i1, 
  i2, 
  i3, 
  i4, 
  i5, 
  i6, 
  o1, 
  o2);


input i1;
input i2;
input i3;
input i4;
input i5;
input i6;
output o1;
output [1:0] o2;

endmodule
\end{lstlisting}
\end{minipage}
\end{minipage}
\hspace{1ex}
\begin{minipage}[t]{.5\textwidth}
\begin{lstlisting}[caption={Wrapper module in \sv{}}, label={lst:modb in sv}]
// declarations...

// /tmp/xin_meng/mhdlc/test/modb.mhdl:2.0-4
moda x_moda (
                .o1 (o1),
                .i1 (i1),
                .i2 (i2),
                .o2 (o2),
                .i3 (i3),
                .i4 (i4),
                .i5 (i5),
                .i6 (i6)
            );

// /tmp/xin_meng/mhdlc/test/modb.mhdl:6.0-21
moda x1_moda (
                 .o1 (x1_o1),
                 .i1 (x1_i1),
                 .i2 (x1_i2),
                 .o2 (x1_o2),
                 .i3 (x1_i3),
                 .i4 (x1_i4),
                 .i5 (x1_i5),
                 .i6 (x1_i6)
             );

// /tmp/xin_meng/mhdlc/test/modb.mhdl:10.0-11.21
moda x2_moda (
                 .o1 (x2_o1_22),
                 .i1 (x2_i1_22),
                 .i2 (x2_i2_22),
                 .o2 (x2_o2_22),
                 .i3 (x2_i3_22),
                 .i4 (x2_i4_22),
                 .i5 (x2_i5_22),
                 .i6 (x2_i6_22)
             );

// /tmp/xin_meng/mhdlc/test/modb.mhdl:15.0-16.27
moda x3_moda (
                 .o1 (out1),
                 .i1 (in1),
                 .i2 (in2),
                 .o2 (out2),
                 .i3 (in3),
                 .i4 (in4),
                 .i5 (in5),
                 .i6 (in6)
             );
\end{lstlisting}
\end{minipage}

\section{Parameterization}\mhdl{} enables designers to creates parameterized module in two ways: 
\begin{itemize}
\item Write parameterized module from draft.
\item Build parameterized module from existing parameterized modules. 
\end{itemize}

Designers declare parameters, and use them in ports or net index. \mhdlc{} 
will automatically parameterize ports in generated declarations. If a module 
to be instantiated is a parameterized module, \mhdlc{} can trace parameter usage
in port connections and automatically parameterize wrapper module. 

%% template part
  \begin{minipage}[t]{.5\textwidth}
    \vspace{0pt}
    \begin{minipage}{\textwidth}
      \begin{lstlisting}[caption={\texttt{modc} in \mhdl{}}, label={lst:modc in mhdl}]
parameter A = 4;
parameter B = 5; 
parameter C = A + B;

assign o1[C-1:0] = {~i1[A-1:0], i2[B-1:0]};
      \end{lstlisting}
    \end{minipage}

    \begin{minipage}{\textwidth}
      \begin{lstlisting}[caption={\texttt{modc} in \sv{}}, label={lst:modc in sv}]
module modc (
  i1, 
  i2, 
  o1);

parameter A = 4;
parameter B = 5;
parameter C = 4 + 5;

input [A - 1:0] i1;
input [B - 1:0] i2;
output [C - 1:0] o1;

logic [A - 1:0] i1;
logic [B - 1:0] i2;
logic [C - 1:0] o1;

// /tmp/xin_meng/mhdlc/test/modc.mhdl:5.0-42
assign o1[C - 1:0] = {~i1[A - 1:0], i2[B - 1:0]};

endmodule
      \end{lstlisting}
    \end{minipage}
  \end{minipage}
  \begin{minipage}[t]{.5\textwidth}
    \vspace{2ex}
    \autoref{lst:modc in mhdl} is a parameterized module to be instantiated. 
    Three parameters are defined in it, in which \texttt{C} depends on other two.

    \autoref{lst:modc in sv} is the generated \sv{} 
    code, all ports are parameterized which fits designers' intend pretty well.
  \end{minipage}


%% instance part
\autoref{lst:modd in mhdl} is a wrapper module named `modd' that 
contains three instantiations of modc with different
parameter settings. `modd' itself is parameterized by two parameters, this 
example demonstrates the automatic parameterization through instantiation.

First instance only overrides value of \texttt{A} via named override. 
Second instance uses positioned override to set values of \texttt{A} and \texttt{B}, 
parameters in wrapper module are used. 
Third instance only overrides value of \texttt{A} via named override, 
parameters in wrapper module are used. 


\begin{minipage}[t]{.45\textwidth}
\begin{lstlisting}[caption={Instantiation}, label={lst:modd in mhdl}]
parameter SETA = 8,
          SETB = 9;


modc #( A = 2 ) x0_modc ( x0_ + );

modc #( SETA, SETB ) x1_modc ( x1_ + );

modc #( A = SETA ) x2_modc (x2_ +,
                           .o1 (x2_o1[10:0]));
\end{lstlisting}

\autoref{lst:modd in sv} is the generated \sv{} from \autoref{lst:modd in mhdl}.

First instance is configured by constant, new nets created by prefix rule are not
parameterized. 

In second instance, \texttt{A} and \texttt{B} are override by parameters in wrapper
module, \texttt{C} is untouched. To preserve the parameter dependency in \texttt{modc}, 
compiler will record parameter usage and propagated the dependency to the wrapper module. 
So port \verb|x1_o1| is parameterized to \verb|x1_o1[SETA+SETB-1]|, which captures designers'
intent perfectly. 

Third instance is a mixture of parameter override and constant override. Besides, port 
\verb|o1| is explicitly connected to \verb|x2_o1[10:0]|. In this case, compiler will not 
parameterize \verb|x2_o1|. 
\end{minipage}
\hspace{1ex}
\begin{minipage}[t]{.45\textwidth}
\begin{lstlisting}[caption={SV Instantiation}, label={lst:modd in sv}]
module modd ( 
// port list...
);

parameter SETA = 8;
parameter SETB = 9;

input [1:0] x0_i1;
input [4:0] x0_i2;
output [6:0] x0_o1;
input [SETA - 1:0] x1_i1;
input [SETB - 1:0] x1_i2;
output [SETA + SETB - 1:0] x1_o1;
input [SETA - 1:0] x2_i1;
input [4:0] x2_i2;
output [10:0] x2_o1;

// variable declarations...

// /tmp/xin_meng/mhdlc/test/modd.mhdl:5.0-33
modc #(
       2,	// A
       5,	// B
       2 + 5	// C
      ) x0_modc (
                 .o1 (x0_o1),
                 .i1 (x0_i1),
                 .i2 (x0_i2)
                );

// /tmp/xin_meng/mhdlc/test/modd.mhdl:7.0-38
modc #(
       SETA,	// A
       SETB,	// B
       SETA + SETB	// C
      ) x1_modc (
                 .o1 (x1_o1),
                 .i1 (x1_i1),
                 .i2 (x1_i2)
                );

// /tmp/xin_meng/mhdlc/test/modd.mhdl:9.0-10.18
modc #(
       SETA,	// A
       5,	// B
       SETA + 5	// C
      ) x2_modc (
                 .o1 (x2_o1[10:0]),
                 .i1 (x2_i1),
                 .i2 (x2_i2)
                );

endmodule
\end{lstlisting}
\end{minipage}


\section{Escape from \mhdl}\mhdl{} provide keywords \texttt{rawcode} and
 \texttt{endrawcode} to escape code from \mhdlc{}.
Designers can write non-\mhdl{} syntax in side this block, 
all contents are copied to generated \sv{} literally.

