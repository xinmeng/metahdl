\mhdl{} enables designers to creates parameterized module in two ways: 
\begin{itemize}
\item Write parameterized module from draft.
\item Build parameterized module from existing parameterized modules. 
\end{itemize}

Designers declare parameters, and use them in ports or net index. \mhdlc{} 
will automatically parameterize ports in generated declarations. If a module 
to be instantiated is a parameterized module, \mhdlc{} can trace parameter usage
in port connections and automatically parameterize wrapper module. 

%% template part
  \begin{minipage}[t]{.5\textwidth}
    \vspace{0pt}
    \begin{minipage}{\textwidth}
      \begin{lstlisting}[caption={\texttt{modc} in \mhdl{}}, label={lst:modc in mhdl}]
parameter A = 4;
parameter B = 5; 
parameter C = A + B;

assign o1[C-1:0] = {~i1[A-1:0], i2[B-1:0]};
      \end{lstlisting}
    \end{minipage}

    \begin{minipage}{\textwidth}
      \begin{lstlisting}[caption={\texttt{modc} in \sv{}}, label={lst:modc in sv}]
module modc (
  i1, 
  i2, 
  o1);

parameter A = 4;
parameter B = 5;
parameter C = 4 + 5;

input [A - 1:0] i1;
input [B - 1:0] i2;
output [C - 1:0] o1;

logic [A - 1:0] i1;
logic [B - 1:0] i2;
logic [C - 1:0] o1;

// /tmp/xin_meng/mhdlc/test/modc.mhdl:5.0-42
assign o1[C - 1:0] = {~i1[A - 1:0], i2[B - 1:0]};

endmodule
      \end{lstlisting}
    \end{minipage}
  \end{minipage}
  \begin{minipage}[t]{.5\textwidth}
    \vspace{2ex}
    \autoref{lst:modc in mhdl} is a parameterized module to be instantiated. 
    Three parameters are defined in it, in which \texttt{C} depends on other two.

    \autoref{lst:modc in sv} is the generated \sv{} 
    code, all ports are parameterized which fits designers' intend pretty well.
  \end{minipage}


%% instance part
\autoref{lst:modd in mhdl} is a wrapper module named `modd' that 
contains three instantiations of modc with different
parameter settings. `modd' itself is parameterized by two parameters, this 
example demonstrates the automatic parameterization through instantiation.

First instance only overrides value of \texttt{A} via named override. 
Second instance uses positioned override to set values of \texttt{A} and \texttt{B}, 
parameters in wrapper module are used. 
Third instance only overrides value of \texttt{A} via named override, 
parameters in wrapper module are used. 


\begin{minipage}[t]{.45\textwidth}
\begin{lstlisting}[caption={Instantiation}, label={lst:modd in mhdl}]
parameter SETA = 8,
          SETB = 9;


modc #( A = 2 ) x0_modc ( x0_ + );

modc #( SETA, SETB ) x1_modc ( x1_ + );

modc #( A = SETA ) x2_modc (x2_ +,
                           .o1 (x2_o1[10:0]));
\end{lstlisting}

\autoref{lst:modd in sv} is the generated \sv{} from \autoref{lst:modd in mhdl}.

First instance is configured by constant, new nets created by prefix rule are not
parameterized. 

In second instance, \texttt{A} and \texttt{B} are override by parameters in wrapper
module, \texttt{C} is untouched. To preserve the parameter dependency in \texttt{modc}, 
compiler will record parameter usage and propagated the dependency to the wrapper module. 
So port \verb|x1_o1| is parameterized to \verb|x1_o1[SETA+SETB-1]|, which captures designers'
intent perfectly. 

Third instance is a mixture of parameter override and constant override. Besides, port 
\verb|o1| is explicitly connected to \verb|x2_o1[10:0]|. In this case, compiler will not 
parameterize \verb|x2_o1|. 
\end{minipage}
\hspace{1ex}
\begin{minipage}[t]{.45\textwidth}
\begin{lstlisting}[caption={SV Instantiation}, label={lst:modd in sv}]
module modd ( 
// port list...
);

parameter SETA = 8;
parameter SETB = 9;

input [1:0] x0_i1;
input [4:0] x0_i2;
output [6:0] x0_o1;
input [SETA - 1:0] x1_i1;
input [SETB - 1:0] x1_i2;
output [SETA + SETB - 1:0] x1_o1;
input [SETA - 1:0] x2_i1;
input [4:0] x2_i2;
output [10:0] x2_o1;

// variable declarations...

// /tmp/xin_meng/mhdlc/test/modd.mhdl:5.0-33
modc #(
       2,	// A
       5,	// B
       2 + 5	// C
      ) x0_modc (
                 .o1 (x0_o1),
                 .i1 (x0_i1),
                 .i2 (x0_i2)
                );

// /tmp/xin_meng/mhdlc/test/modd.mhdl:7.0-38
modc #(
       SETA,	// A
       SETB,	// B
       SETA + SETB	// C
      ) x1_modc (
                 .o1 (x1_o1),
                 .i1 (x1_i1),
                 .i2 (x1_i2)
                );

// /tmp/xin_meng/mhdlc/test/modd.mhdl:9.0-10.18
modc #(
       SETA,	// A
       5,	// B
       SETA + 5	// C
      ) x2_modc (
                 .o1 (x2_o1[10:0]),
                 .i1 (x2_i1),
                 .i2 (x2_i2)
                );

endmodule
\end{lstlisting}
\end{minipage}

