Preprocessor helps designers to embed script like 
code configuration directives into RTL code for reuse 
oriented designs to improve code integrity. Conventionally, 
designers are used to write one-time scripts (Perl/sed/awk/csh)
to preprocess their RTL for similar project usage. This methodology
is not clean enough. Verification engineers have to create additional
steps in \texttt{Makefile} to preprocess code. 
\mhdl{}'s preprocessor uses \vlog{} style macro syntax, introduces 
more flow control directives that help designers perform conditional and 
repetitive configuration on RTL. 

In addition to conventional \`{}\texttt{ifdef}, \`{}\texttt{ifndef}, 
\`{}\texttt{else}, \`{}\texttt{define} and \`{}\texttt{include} 
macro directives, \mhdl{} introduces \`{}\texttt{for}, \`{}\texttt{if} and
\`{}\texttt{let} to enlarge the power of preprocessor (see following examples). 

\autoref{lst:arb in pp} is a simple Round Robin Arbiter FSM implemented 
in \mhdl{} with facilitating preprocessor. This arbiter can respond to 
a configurable number of slaves, which is controlled by macro \texttt{SLV\_NUM}. 
Once the \mhdl{} code is finished, various arbiters can be generated in \sv{} with 
giving different values to \texttt{SLV\_NUM} when invoke \mhdlc{}.

\begin{minipage}[t]{.45\textwidth}
\begin{lstlisting}[caption={Configurable Arbiter}, label={lst:arb in pp}]
fsm arb; 

/*-\`{}\textcolor{purple}{for}-*/ (i=1; /*-\`{}\textcolor{purple}{i}-*/<=/*-\`{}\textcolor{purple}{SLV\_NUM}-*/; i++) /*-\label{ln:default output}-*/
slave_grnt_/*-\`{}\textcolor{purple}{i}-*/ = 1'b0;
/*-\`{}\textcolor{purple}{endfor}-*/

/*-\`{}\textcolor{purple}{for}-*/ (i=1; /*-\`{}\textcolor{purple}{i}-*/<=/*-\`{}\textcolor{purple}{SLV\_NUM}-*/; i++) /*-\label{ln:body}-*/
 /*-\`{}\textcolor{purple}{let}-*/ j = /*-\`{}\textcolor{purple}{i}-*/ + 1 /*-\label{ln:let}-*/

 /*-\`{}\textcolor{purple}{if}-*/ /*-\`{}\textcolor{purple}{i}-*/ != /*-\`{}\textcolor{purple}{SLV\_NUM}-*/ /*-\label{ln:if}-*/
SLAVE_/*-\`{}\textcolor{purple}{i}-*/: begin
   if ( slave_req_/*-\`{}\textcolor{purple}{i}-*/ ) begin
      slave_grnt_/*-\`{}\textcolor{purple}{i}-*/ = 1'b1;
      if ( slave_eof_/*-\`{}\textcolor{purple}{i}-*/ ) begin
	 slave_grnt_/*-\`{}\textcolor{purple}{i}-*/ = 1'b0;
	 goto SLAVE_/*-\`{}\textcolor{purple}{j}-*/;
      end
      else begin
	 goto SLAVE_/*-\`{}\textcolor{purple}{i}-*/;
      end
   end
   else 
     goto SLAVE_/*-\`{}\textcolor{purple}{j}-*/;
end

 /*-\`{}\textcolor{purple}{else} \label{ln:else}-*/ 
SLAVE_/*-\`{}\textcolor{purple}{i}-*/: begin 
   if ( slave_req_/*-\`{}\textcolor{purple}{i}-*/ ) begin
      slave_grnt_/*-\`{}\textcolor{purple}{i}-*/ = 1'b1;
      if ( slave_eof_/*-\`{}\textcolor{purple}{i}-*/ ) begin
	 slave_grnt_/*-\`{}\textcolor{purple}{i}-*/ = 1'b0;
	 goto SLAVE_1;
      end
      else 
	goto SLAVE_/*-\`{}\textcolor{purple}{i}-*/;
   end
   else 
     goto SLAVE_1;
end  
 /*-\`{}\textcolor{purple}{endif} \label{ln:endif}-*/
/*-\`{}\textcolor{purple}{endfor}-*/
endfsm
\end{lstlisting}
\end{minipage}
\begin{minipage}[t]{.5\textwidth}
\vspace{2ex}
Line \autoref{ln:default output} starts a \`{}\texttt{for} directive to repetitively 
``write'' code with slight difference. Default values of FSM output are set within 
this block. 

Line \autoref{ln:body} starts another \`{}\texttt{for} directive to write slave handling
code, one state for each slave. 
Since it is a Round Robin arbiter, every states perform same task: grant slave access if
has request, move to next slave when current one has no request or transaction is done, 
roll back to the first slave when a arbitration round finishes. Line \autoref{ln:if}, 
\autoref{ln:else} and \autoref{ln:endif} compose an \`{}\texttt{if} block to check
whether current state is for last slave. 

Line \autoref{ln:let} is \`{}\texttt{let} directive used to perform arithmetic operation 
and calculate value of \`{}\texttt{j}, which is the number of next slave.

\autoref{lst:arb in sv} is the generated \sv{} with \texttt{SLV\_NUM} set to $4$. 
\end{minipage}

\begin{lstlisting}[caption={4 slave arbiter \sv{} code}, label={lst:arb in sv}]
module arbiter (
  clock, 
  reset_n, 
  slave_eof_1, 
  slave_eof_2, 
  slave_eof_3, 
  slave_eof_4, 
  slave_grnt_1, 
  slave_grnt_2, 
  slave_grnt_3, 
  slave_grnt_4, 
  slave_req_1, 
  slave_req_2, 
  slave_req_3, 
  slave_req_4);


input clock;
input reset_n;
input slave_eof_1;
input slave_eof_2;
input slave_eof_3;
input slave_eof_4;
output slave_grnt_1;
output slave_grnt_2;
output slave_grnt_3;
output slave_grnt_4;
input slave_req_1;
input slave_req_2;
input slave_req_3;
input slave_req_4;

const logic [3:0] SLAVE_1 = 4'b0001;
const logic [3:0] SLAVE_2 = 4'b0010;
const logic [3:0] SLAVE_3 = 4'b0100;
const logic [3:0] SLAVE_4 = 4'b1000;
const int _SLAVE_1_ = 0;
const int _SLAVE_2_ = 1;
const int _SLAVE_3_ = 2;
const int _SLAVE_4_ = 3;
logic [3:0] arb_cs;
logic [3:0] arb_ns;
logic clock;
logic reset_n;
logic slave_eof_1;
logic slave_eof_2;
logic slave_eof_3;
logic slave_eof_4;
logic slave_grnt_1;
logic slave_grnt_2;
logic slave_grnt_3;
logic slave_grnt_4;
logic slave_req_1;
logic slave_req_2;
logic slave_req_3;
logic slave_req_4;

// Sequential part of FSM /tmp/xin_meng/mhdlc/test/arbiter.mhdl:1.0-42.5
// /tmp/xin_meng/mhdlc/test/arbiter.mhdl:1.0-42.5
always_ff @(posedge clock or negedge reset_n)
  if (~reset_n) begin
    arb_cs <= SLAVE_1;
  end
  else begin
    arb_cs <= arb_ns;
  end

// Combnational part of FSM /tmp/xin_meng/mhdlc/test/arbiter.mhdl:1.0-42.5
// /tmp/xin_meng/mhdlc/test/arbiter.mhdl:1.0-42.5
always_comb begin
  slave_grnt_1 = 1'b0;
  slave_grnt_2 = 1'b0;
  slave_grnt_3 = 1'b0;
  slave_grnt_4 = 1'b0;
  unique case ( 1'b1 )
    arb_cs[_SLAVE_1_] : begin
      if ( slave_req_1 ) begin
        slave_grnt_1 = 1'b1;
        if ( slave_eof_1 ) begin
          slave_grnt_1 = 1'b0;
          arb_ns = SLAVE_2;
        end
        else begin
          arb_ns = SLAVE_1;
        end
      end
      else begin
        arb_ns = SLAVE_2;
      end
    end

    arb_cs[_SLAVE_2_] : begin
      if ( slave_req_2 ) begin
        slave_grnt_2 = 1'b1;
        if ( slave_eof_2 ) begin
          slave_grnt_2 = 1'b0;
          arb_ns = SLAVE_3;
        end
        else begin
          arb_ns = SLAVE_2;
        end
      end
      else begin
        arb_ns = SLAVE_3;
      end
    end

    arb_cs[_SLAVE_3_] : begin
      if ( slave_req_3 ) begin
        slave_grnt_3 = 1'b1;
        if ( slave_eof_3 ) begin
          slave_grnt_3 = 1'b0;
          arb_ns = SLAVE_4;
        end
        else begin
          arb_ns = SLAVE_3;
        end
      end
      else begin
        arb_ns = SLAVE_4;
      end
    end

    arb_cs[_SLAVE_4_] : begin
      if ( slave_req_4 ) begin
        slave_grnt_4 = 1'b1;
        if ( slave_eof_4 ) begin
          slave_grnt_4 = 1'b0;
          arb_ns = SLAVE_1;
        end
        else begin
          arb_ns = SLAVE_4;
        end
      end
      else begin
        arb_ns = SLAVE_1;
      end
    end

    default: begin
      arb_ns = 4'hX;
    end
  endcase
end


endmodule
\end{lstlisting}



\`{}\texttt{let} directive supports sufficient operators and a bunch of common used functions:
\begin{itemize}
\item Numeric operators: addition (+), subtraction ($-$),  multiplication ($*$), division ($/$), 
modulus (\%), power ($**$).
\item Logical operators: logical AND (\&\&), logical OR ($||$), logical NOT ($!$).
\item Bit operators: bitwise XOR (\^{}), bitwise AND  (\&),  bitwise OR ($|$), 
shift right ($>>$), shift left ($<<$).
\item Functions: log 2 (LOG2()), round up (CEIL()), round down (FLOOR()), 
round to nearest value (ROUND()),  max of two numbers (MAX()), 
min of two numbers (MIN()), odd (ODD()), even (EVEN()), absolute value (ABS()).
\end{itemize}

\begin{lstlisting}[caption={\`{}\texttt{let} usage examples},label={lst:let example}]
/*-\`{}\textcolor{purple}{define}-*/ x 2

// NOTE /*-\`{}\textcolor{purple}{let}-*/ need "="
/*-\`{}\textcolor{purple}{let}-*/ y = /*-\`{}\textcolor{purple}{x}-*/ ** 10 // now /*-\`{}\textcolor{purple}{y}-*/ is 1024

/*-\`{}\textcolor{purple}{let}-*/ z = LOG2(/*-\`{}\textcolor{purple}{y}-*/) // /*-\`{}\textcolor{purple}{z}-*/ is 10
/*-\`{}\textcolor{purple}{let}-*/ a = LOG2(/*-\`{}\textcolor{purple}{z}-*/) // /*-\`{}\textcolor{purple}{a}-*/ is 3.321928
/*-\`{}\textcolor{purple}{let}-*/ c = CEIL(/*-\`{}\textcolor{purple}{a}-*/) // /*-\`{}\textcolor{purple}{c}-*/ is 4
/*-\`{}\textcolor{purple}{let}-*/ f = FLOOR(/*-\`{}\textcolor{purple}{a}-*/) // /*-\`{}\textcolor{purple}{f}-*/ is 3
/*-\`{}\textcolor{purple}{let}-*/ r = ROUND(/*-\`{}\textcolor{purple}{a}-*/) // /*-\`{}\textcolor{purple}{r}-*/ is 3


// we can concatenate macro value with other strings
// using "::" operator
assign cat_/*-\`{}\textcolor{purple}{f}-*/::k = 1'b0; // expand to ``assign cat_3k = 1'b0;''
assign cat_/*-\`{}\textcolor{purple}{fk}-*/ = 1'b0; // /*-\textcolor{red}{Error!!}-*/ macro ``fk'' is not defined 
\end{lstlisting}
\autoref{lst:let example} demonstrates some usages of \`{}\texttt{let} directive. 
Operators and functions supported by \`{}\texttt{let} directive constructs macro variables
to \emph{macro expression}, such as 
\mbox{\texttt{\`{}a + \`{}b}}, 
\mbox{\texttt{ODD(\`{}f)}}, 
\mbox{\texttt{\`{}a >> 2}}, 
\mbox{\texttt{\`{}f <= \`{}d \&\& \`{}f != 0}},
which can further be evaluated by \`{}\texttt{if} directive for flow controlling, 
just as shown in \autoref{lst:arb in pp} line \autoref{ln:if} . 
