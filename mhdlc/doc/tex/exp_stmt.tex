Just as \vlog{}, \sv{} or C/C++, \emph{expression}\index{expression}
is the bottom level language elements, it
constructs \emph{statement}\index{statement}, statement then
constructs code blocks\index{code block} which ultimately constructs
module. Expression is the same as that in synthesizable \vlog{}
or \sv{}. Expression is recursively defined in BNF, prime expressions
are net and constant, which leads out of the recursion.

Statement is built upon expression through assignment, conditional
control or repetition. Empty statement is also allowed in \mhdl{},
since it does not make any sense, warning is popped when empty
statement is encountered. `if-else', `case', `for' statements are
supported, `goto' statement is newly added for state transition (refer
to \autoref{sec:syntax.fsm}).

Since \mhdl{} inherits \sv{} syntax, rules for expression and
statements are same as that in \sv{}, refer to \autoref{sec:bnf} for
complete rule list. 
