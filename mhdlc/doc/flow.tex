\insertfigure{\mhdl{} compilation flowchart}{mflow.pdf}{fig:mflow}
\autoref{fig:mflow} is \mhdl{} compilation flow. \mhdlc{} has a
build-in directory based automatic dependency resolving capability,
which means files to be processed can be supplied in any
order\footnote{ More strictly, macro definition files should be
  provided before macro usage files, otherwise, because compiler can
  not expand macro without definition.  }, or only provide top level
wrapper module, compiler can find all instantiated modules in search
path which is specified by user through command line.

Once the module is compiled, module definition is stored in module
database maintained by compiler. When module instantiation statement is
encountered, compiler first searches module definition in module
database, if not found, compiler start automatic dependency resolving
to find module definition. 

Automatic dependency resolving is
\emph{comprehensive}\index{Comprehensive dependency resolving} because compiler will
search \mhdl{}, \vlog{} or \sv{} format module definitions in search
path. Multiple definition is considered to be fatal error, and
designers are responsible to fix the problem. Comprehensive dependency
resolving requires traversing of search path three times for each
module, this could cost very long runtime especially when search path
list is very large. \emph{Fast dependency resolving}\index{Fast
  dependency resolving} mode can be enabled to save time. In this
mode, first found module definition is used, designers are in their
own risk of multiple definition. This mode can achieve 2x-4x faster
than comprehensive mode. 

\mhdlc{} puts all generated files in centralized output directory
(default name is ``workdir'')
which can be specified by users. RTL are generated in \sv{} format
with ``.sv'' as file extension. Files with ``.postpp'' extension are
output files from preprocessor, only used for compiler
debugging. Normally, only \sv{} files generated from \mhdl{} are put
in output directory, \vlog{} or \sv{} parsed by compiler are
\emph{not} copied to output directory. However, users can still ask
compiler to put all touched \vlog{} or \sv{} files into output
directory with command line options. 
