\mhdl{} provides control syntax start with keyword ``\texttt{metahdl}'', 
which interfaces with \mhdlc{} and controls the runtime behavior of compiler. 
Designers' controls are passed to compiler via variable assignments embedded 
in RTL code, this variable settings are also preceded by keyword \texttt{metahdl}.
Boolean variables inside compiler are set via $+$/$-$ preceded by variable name, 
where $+$ means ``enable'' and $-$ means ``disable''. 
There are two special form of control syntax: exit syntax, and echo syntax. 
The former is used to command compiler exit when the statement is encountered. 
The latter is used to print messages on \texttt{stderr}. They are usually used 
with preprocessor to guarantee correct configuration settings. 
Refer to \autoref{sec:bnf} for detailed formal syntax.

Working scope of all variables can be \emph{Modular}\index{modular variable} or \emph{Effective}\index{effective variable}. 
Modular variables (MVAR)\index{MVAR} take effect on entire module and are used when parsing is finished. 
Designers can set MVAR anywhere in source code and get the same 
effect. If an MVAR is assigned multiple times, last assignment wins. 
MVAR can have different values in different files, so file is
the minimum granularity of MVAR.

Effective variables (EVAR)\index{EVAR} take effect from the point the variable is assigned
and are used \emph{during} parsing. Designers can set different values for same EVAR in different
sections of source code, and make compiler treat sections differently. So the minimum granularity 
of EVAR is section divided by EVAR assignments.

\section{Variable List}
Following is the complete list of all compiler variables can be assigned by user control 
syntax, variable type (boolean or string) and variable scope (MVAR or EVAR)  are listed
with variable name. 

\varline{modname}{MVAR}{string}{Base file name}{
Set the generated module name. Often used with preprocessor to 
distinguish modules with different configurations. 
}

\varline{outfile}{MVAR}{string}{Base file name}{
Set the generated \sv{} file base name. Often used with preprocessor to 
distinguish module definition files with different configurations. }

\varline{portchk}{MVAR}{boolean}{false}{
Enable/Disable port validation for module.
}

\varline{hierachydepth}{MVAR}{positive int}{300}{
Maximum level of module instantiation. 
}

\varline{clock}{EVAR}{string}{clock}{
Default clock name used for \texttt{ff\_block} and \texttt{fsm\_block}.
}

\varline{reset}{EVAR}{string}{reset\_n}{
Default reset name used for \texttt{ff\_block} and \texttt{fsm\_block}.
}

\varline{multidriverchk}{MVAR}{boolean}{true}{
Enable/Disable multiple driver checking for module. 
}

\varline{relexedfsm}{EVAR}{boolean}{true}{
Set severity of connectivity/reachability error checked in FSM. 
If it is true, relaxed FSM programming mode is enabled, 
all dead states or unreachable states are acceptable, compiler
only reports warning when such states are encountered, and continues processing.  
if it is false, FSM programming is in strict mode, any dead state or
unreachable state is considered to be fatal error, compiler will report error
and stop processing if such state is checked. 
}

\varline{exitonwarning}{EVAR}{boolean}{false}{
Set severity of \emph{normal parsing warning}, such as width mismatch. 
If it is true, compiler exits on any warning. 
}

\varline{exitonlintwarning}{MVAR}{boolean}{false}{
Warning from port validation, multiple driver checking are categorized 
as lint warning. If this variable is set to true, compiler will exit
on any lint warning. 
}

\section{Example}
\autoref{lst:top wrapper} demonstrates  user control syntax with 
code configuration. 

\begin{minipage}[t]{.45\textwidth}
\begin{lstlisting}[caption={Demo Top Wrapper}, label={lst:top wrapper}]
metahdl + portchk; /*\label{ln:portchk}*/

/*\`{}\textcolor{purple}{if}*/ WIDTH > 64 /*\label{ln:chk width}*/
metahdl ``width can not exceed 64!'';
metahdl exit;
/*\`{}\textcolor{purple}{endif}*/

assign data[/*\`{}\textcolor{purple}{WIDTH}*/-1:0] = /*\`{}\textcolor{purple}{WIDTH}*/'d0;

/*\`{}\textcolor{purple}{ifdef}*/ FPGA
/*\`{}\textcolor{purple}{define}*/ target fpga
/*\`{}\textcolor{purple}{else}*/
/*\`{}\textcolor{purple}{define}*/ target asic
/*\`{}\textcolor{purple}{endif}*/

metahdl modname = top_/*\`{}\textcolor{purple}{target}*/; /*\label{ln:modname}*/
metahdl outfile = top_/*\`{}\textcolor{purple}{target}*/; /*\label{ln:outfile}*/

metahdl + exitonwarning; /*\label{ln:exitonwarning on}*/

metahdl clock = clk_125M; 
metahdl reset = pclk_rst_n; 
 /*\label{ln:sec1 b}*/
ff;
  a_ff, a, 1'b0; 
  b_ff[1:0], b, 1'b0;
endff
/*\label{ln:sec1 e}*/
metahdl - exitonwarning; /*\label{ln:exitonwarning off}*/

metahdl clock = clk_250M;
metahdl reset = dclk_rst_n;
/*\label{ln:sec2 b}*/
ff; 
  c_ff, c, 1'b0;
  d_ff, d, 1'b0;
endff

ff; 
  e_ff, e, 1'b0;
  g_ff, g, 1'b0;
endff /*\label{ln:sec2 e}*/
\end{lstlisting}
\end{minipage}
\begin{minipage}[t]{.5\textwidth}
\vspace{2ex}
Line \autoref{ln:portchk} enables port validation 
in this module. 

Line \autoref{ln:chk width} checks value of macro \texttt{WIDTH}, 
forces compilation exit upon illegal values. 

Line \autoref{ln:modname} and \autoref{ln:outfile} alter \sv{} module name and
output file name according to target device. 

Any warning between \autoref{ln:exitonwarning on} and \autoref{ln:exitonwarning off} makes 
compiler exit. To be more specific, width mismatch between \verb|b_ff| and \verb|b| is considered
to be fatal error. 

Different clock and reset names are used for different code sections, section \autoref{ln:sec1 b}-\autoref{ln:sec1 e}, 
section \autoref{ln:sec2 b}-\autoref{ln:sec2 e}.
\end{minipage}
